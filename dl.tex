% Created 2018-02-05 Mon 20:57
\documentclass[11pt]{article}
\usepackage[utf8]{inputenc}
\usepackage[T1]{fontenc}
\usepackage{fixltx2e}
\usepackage{graphicx}
\usepackage{longtable}
\usepackage{float}
\usepackage{wrapfig}
\usepackage{rotating}
\usepackage[normalem]{ulem}
\usepackage{amsmath}
\usepackage{textcomp}
\usepackage{marvosym}
\usepackage{wasysym}
\usepackage{amssymb}
\usepackage{hyperref}
\tolerance=1000
\author{Daniel Guimaraes}
\date{\today}
\title{Description Logic}
\hypersetup{
  pdfkeywords={},
  pdfsubject={},
  pdfcreator={Emacs 25.1.1 (Org mode 8.2.10)}}
\begin{document}

\maketitle
\tableofcontents

\section{7 - Query Answering}
\label{sec-1}
\subsection{Query Answering in Databases \\}
\label{sec-1-1}
\subsubsection{Database \\}
\label{sec-1-1-1}
    A finite collection of related data which has some inherent 
meaning. The main difference between data and knowledge bases is that 
while the former concentrate on \textit\{manipulating large and 
persistent models of relatively simple data\}, the latter provide 
more support for \textit{inference—finding answers} about the model 
which had not been explicitly told to it—and involve fewer but more 
complex data. In DL database can bes seen as a finite interpretation
under a domain $\Delta^{\mathcal{I}}$ together with relations 
interpreting the relation symbols.
\subsubsection{SQL query \\}
\label{sec-1-1-2}
    We restrict the attention to unary and binary relation symbols 
corresponding to concepts and roles in DLs. We give the definitions
for arbitrary interpretations not only finite ones.
    Describe answer tuples, same expressiveness as FO formula, with 
free variables (answer variables).
\begin{enumerate}
\item Def 7.1: (First Order Query) \\
\label{sec-1-1-2-1}
     Is a first order formula that uses only unary and binary 
predicates (concept and role names), and no function symbols or 
constants. The use of equality is allowed!\\
     The free variables $\overset{\to}{x}$ of an FO query 
$q(\overset{\to}{x})$ are called answer variables. The arity of 
$q(\overset{\to}{x})$ is the number of answer variables.\\
Let $q(\overset{\to}{x})$ be a FO query of arity k, and $\mathcal{I}$
an interpretation. We say that:
\begin{center}
$q(\overset{\to}{a})$ = $a_{1}, ..., a_{k}$ is an answer to q on 
$\mathcal{I}$ if $\mathcal{I} \models q[\overset{\to}{a}]$\\
\end{center}
i.e. if $q[\overset{\to}{a}]$ evaluates to true in $\mathcal{I}$ under 
the valuation that interprets the answer variables $\overset{\to}{a}$ 
to the constants $a_{1}, ..., a_{k}$. The set of all answers to q 
on $\mathcal{I}$ can be denoted as $ans(q,\mathcal{I})$ 

\item Def 7.2 (conjuntive query):\\
\label{sec-1-1-2-2}
     A conjunctive query q has the form:\\
     $\exists$ x$_{\text{1}}$, \ldots{}, $\exists$ x$_{\text{k}}$ $\alpha$$_{\text{1}}$ \and \ldots{} 
     $\alpha$$_{\text{n}}$\\
     where k >= 0, n >=1 and $\alpha_{i}$ is a concept atom $A(x)$ or a 
role atom $r(x,y)$ with $A \in \mathbf(C)$, $r \in \mathbf{R}$, and 
x,y are variables. The xs are que quantified variables and the rest
are the answered variables.\\

\item Def 7.3 ($\overset{\to}{a}-match$):\\
\label{sec-1-1-2-3}
Let q be a conjunctie query and $\mathcal{I}$ an interpretation.\\
     We use $var(q)$ to denote the set of all variables in q.\\

A match of q in $\mathcal{I}$ is a mapping: $\pi: var(q) \to 
\Delta^{\mathcal{I}}$ such that:
\begin{itemize}
  \item $\pi(x) \in A^{\mathcal{I}}$ for all concept atoms $A(x)$ in q
  \item $(\pi(x),\pi(y)) \in r^{\mathcal{I}}$ for all role atoms 
r(x,y) in q.
\end{itemize}
\end{enumerate}








\subsection{Ontology-mediated Query Answering}
\label{sec-1-2}
% Emacs 25.1.1 (Org mode 8.2.10)
\end{document}
