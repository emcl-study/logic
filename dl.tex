% Created 2018-02-06 Tue 17:56
\documentclass[11pt]{article}
\usepackage[utf8]{inputenc}
\usepackage[T1]{fontenc}
\usepackage{fixltx2e}
\usepackage{graphicx}
\usepackage{longtable}
\usepackage{float}
\usepackage{wrapfig}
\usepackage{rotating}
\usepackage[normalem]{ulem}
\usepackage{amsmath}
\usepackage{textcomp}
\usepackage{marvosym}
\usepackage{wasysym}
\usepackage{amssymb}
\usepackage{hyperref}
\tolerance=1000
\author{Daniel Guimaraes}
\date{\today}
\title{Description Logic}
\hypersetup{
  pdfkeywords={},
  pdfsubject={},
  pdfcreator={Emacs 25.1.1 (Org mode 8.2.10)}}
\begin{document}

\maketitle
\tableofcontents

\section{7 - Query Answering}
\label{sec-1}
\subsection{Query Answering in Databases \\}
\label{sec-1-1}
\subsubsection{Database \\}
\label{sec-1-1-1}
A finite collection of related data which has some inherent 
meaning. The main difference between data and knowledge bases is that 
while the former concentrate on \textit\{manipulating large and 
persistent models of relatively simple data\}, the latter provide 
more support for \textit{inference—finding answers} about the model 
which had not been explicitly told to it—and involve fewer but more 
complex data. In DL database can bes seen as a finite interpretation
under a domain $\Delta^{\mathcal{I}}$ together with relations 
interpreting the relation symbols.
\subsubsection{SQL query \\}
\label{sec-1-1-2}
We restrict the attention to unary and binary relation symbols 
corresponding to concepts and roles in DLs. We give the definitions
for arbitrary interpretations not only finite ones.
Describe answer tuples, same expressiveness as FO formula, with 
free variables (answer variables).
\begin{enumerate}
\item Def 7.1: (First Order Query) \\
\label{sec-1-1-2-1}
Is a first order formula that uses only unary and binary 
predicates (concept and role names), and no function symbols or 
constants. The use of equality is allowed!\\
     The free variables $\overset{\to}{x}$ of an FO query 
$q(\overset{\to}{x})$ are called answer variables. The arity of 
$q(\overset{\to}{x})$ is the number of answer variables.\\
     Let $q(\overset{\to}{x})$ be a FO query of arity k, and $\mathcal{I}$
an interpretation. We say that:
\begin{center}
$q(\overset{\to}{a})$ = $a_{1}, ..., a_{k}$ is an answer to q on 
$\mathcal{I}$ if $\mathcal{I} \models q[\overset{\to}{a}]$\\
\end{center}
i.e. if $q[\overset{\to}{a}]$ evaluates to true in $\mathcal{I}$ under 
the valuation that interprets the answer variables $\overset{\to}{a}$ 
to the constants $a_{1}, ..., a_{k}$. The set of all answers to q 
on $\mathcal{I}$ can be denoted as $ans(q,\mathcal{I})$ 

\item Def 7.2 (conjuntive query):\\
\label{sec-1-1-2-2}
A conjunctive query q has the form:\\
\begin{center}
$\exists x_{1}, ..., \exists x_{k} \alpha_{1} \and ... \alpha_{n}$\\
\end{center}
where k $\ge$ 0, n $\ge$ 1 and $\alpha_{i}$ is a concept atom $A(x)$ or a 
role atom $r(x,y)$ with $A \in \mathbf(C)$, $r \in \mathbf{R}$, and 
x,y are variables. The xs are que quantified variables and the rest
are the answered variables.\\
     Note: A conjunctive query is a special case of FO queries.
\item Def 7.3 ($\overset{\to}{a}-match$): \\
\label{sec-1-1-2-3}
Let q be a conjunctive query and $\mathcal{I}$ an interpretation.\\
     We use $var(q)$ to denote the set of all variables in q.
A match of q in $\mathcal{I}$ is a mapping: $\pi: var(q) \to 
     \Delta^{\mathcal{I}}$ such that:
\begin{itemize}
\item $\pi(x) \in A^{\mathcal{I}}$ for all concept atoms $A(x)$ in q
\item $(\pi(x),\pi(y)) \in r^{\mathcal{I}}$ for all role atoms 
r(x,y) in q.
\end{itemize}
Where $\overset{\to}{x} = x_{1}, ... x_{k}$ be the answer variables
in q and $\overset{\to}{a} = a_{1}, ... a_{k}$ are individual names 
from $\mathbf{I}$.\\
     We call a match $\pi$ of q in $\mathcal{I}$ an $\overset{\to}{a}$-match
if $\pi(x_{i})$ = a$_{\text{i}}^{\mathcal{I}}$ for 1 $\le$ i $\le$ k.

\item Lemma 7.4:\\
\label{sec-1-1-2-4}
$ans(q,\mathcal{I})$ = $\{ \overset{\to}{a} |$ there is an 
$\overset{\to}{a}-match$ of q in $\mathcal{I} \}$
\end{enumerate}

\subsubsection{Complexity}
\label{sec-1-1-3}
\begin{enumerate}
\item Def 7.5 (query entailment)
\label{sec-1-1-3-1}
Let q be a conjuctive query of arity k, $\mathcal{I}$ an 
interpretation and $\overset{\to}{a} = a_{1}, ...a_{k}$. \\
     We say that $\mathcal{I} \models q(\overset{\to}{a})$ 
if $\overset{\to}{a} \in ans(q,\mathcal{I})$.
If k=0, then we call $q$ a Boolean query and simply write 
$\mathcal{I} \models q$.
\item Proposition 7.6:The query entailment problem for conjunctive queries is
\label{sec-1-1-3-2}
NP-complete.
\item Conjunctive query reduction to 3-colorability.
\label{sec-1-1-3-3}
3-colorability a well known NP-complete problem.
The (undirected) graph G = (V,E) is 3colorable if there is a 
mapping $c:V \to \{red,blue,green\}$ such $\{u,v\} \in V$ implies 
$c(u) \ne c(v)$. \\
\begin{tabular}{ l | l }
\textbf{Conjunctive query}  & \textbf{Interpretation $\mathcal{I}$} \\ 
$\exists x_{1},x_{2},x_{3},x_{4},x_{5},x_{6}$. &  $\Delta^{\mathcal{I}} = \{red,blue,green\}$ \\
$E(x_{1},x_{2}) \land E(x_{2},x_{3})$ & $E^{\mathcal{I}} = \{(red,blue), (blue,red)$ \\
$E(x_{1},x_{4}) \land E(x_{2},x_{5}) \land E(x_{3},x_{6})$ & $(red,green), (green,red)$\\
$E(x_{4},x_{5}) \land E(x_{5},x_{6})$ & $(green,blue), (blue,green)\}$ \\
\end{tabular}

\begin{center}
$\mathcal{I} \models$ q iff G is 3-colorable\\
\end{center}
\end{enumerate}

\subsection{Ontology-mediated Query Answering \\}
\label{sec-1-2}
In OMQA we consider:
\begin{itemize}
\item A TBox $\mathcal{T}$ represents \textit{background knowledge}
\item An ABox $\mathcal{A}$ that gives an \textit{incomplete description}
of the data
\item a \textit{a conjunctive query}
\end{itemize}

The \textit{actual data}(the interpretation $\mathcal{I}$) is not 
known, all we know is that it is consistent with $\mathcal{T}$ and 
$\mathcal{A}$, i.e. $\mathcal{I}$ is a model of $\mathcal{T} \cup  
   \mathcal{A}$.
We want to find answers to q which are true to all possible data,
i.e. for all models of $\mathcal{T} \cup \mathcal{A}$:
\begin{center}
\textbf{Certain Answers}
\end{center}

\subsubsection{Certain Answers}
\label{sec-1-2-1}
\begin{enumerate}
\item Def 7.9 (certain answer) \\
\label{sec-1-2-1-1}
Let $\mathcal{K} = (\mathcal{A}, \mathcal{T})$ be a knowledge base
Then $\overset{\to}{a}$ is a certain answer to q on $\mathcal{K}$ if
\begin{itemize}
\item all individual names from $\overset{\to}{a}$ occur in 
$\mathcal{A}$
\item $\overset{\to}{a} \in ans(q, \mathcal{I})$ for every model
$\mathcal{I}$ of $\mathcal{K}$
\end{itemize}
cert(q,$\mathcal{K}$) denotes the set of all certain answers to q
on $\mathcal{K}$, i.e: \\
\begin{center}
$cert(q,\mathcal{K})$ = $\bigcap_{\mathcal{I} \ model \ of \ \mathcal{K}} ans(q, \mathcal{I})$
\end{center}

$\overset{\to}{a} \in cert(q,\mathcal{K})$ iff 
$\mathcal{T} \cup \mathcal{A} \models q(\overset{\to}{a})$.
\end{enumerate}

\subsubsection{Complexity}
\label{sec-1-2-2}
\begin{enumerate}
\item Def 7.10 (OMQA query entailment) \\
\label{sec-1-2-2-1}
Let q be a conjunctive query of arity k, $\mathcal{K}$ a knowledge
base and $\overset{\to}{a} = a_{1}, ... a_{k}$ a tuple of individuals
occuring in $\mathcal{K}$.
We say that $\mathcal{K} \models q(\overset{\to}{a})$ if 
$\overset{\to}{a} \in cert(q,\mathcal{K})$. If k = 0, then we 
simply write $\mathcal{K} \models q$.
\item Data complexity \\
\label{sec-1-2-2-2}
Consider only simple ABoxes, whose assertions are of the form
a:A and (a,b):r where $A\in \mathbf{C}$ and $r \in \mathbf{R}$.
Measure the complexity in the size of the ABox only, and
assume that the Tbox and the query have constant size.\\
     The complexity of OMQA query entailment of course depends on the 
query language and which DL for formulating the KB are used.
\begin{enumerate}
\item Query Language \\
\label{sec-1-2-2-2-1}
We consider only conjunctive queries.
In fact, for FO queries, OMQA query entailment would be 
undecidable. It is not hard to reduce satisfiability of FO formulas
to answering FO queries on a knowledge bases (with an empty TBox)
and thus answers to FO queries on knowledge bases are uncomputable.
\item Proposition 7.11 (data complexity of OMQA in $\mathcal{A}\mathcal{L}\mathcal{C}$) \\
\label{sec-1-2-2-2-2}
  In $\mathcal{A}\mathcal{L}\mathcal{C}$ the query entailment
  problem for conjunctive queries is coNP-hard w.r.t. \textit{data complexity}. 
  TBox and the query are constant, the input graph is translated 
  into the ABox $\mathcal{A}_{G}$ = $\{(u,v):r | \{u,v\}\in E\}$
  We have that $(\mathcal{T},\mathcal{A_{G}}) \models q$ 
  iff G is not 3-colorable.
  Proof:\\
     (=>) Assume that G is 3-colorable and let $c:V \to \{red,blue,green\}$
be the coloring. Let $\mathcal{I}_{c}$ be the interpretation with:\\
     $\Delta$$^{\mathcal{I}_{\text{c}}}$ = V, $v^{\mathcal{I}_{c}}$ = v for all $v \in V$. \\
     r$^{\mathcal{I}_{c}}$ = \$\{(u,v) | \{u,v\} $\in$ E\}\$\\
     R$^{\mathcal{I}_{c}}$ = \{v $\in$ V | c(v)=red \}\\
     B$^{\mathcal{I}_{c}}$ = \{v $\in$ V | c(v)=blue \}\\
     G$^{\mathcal{I}_{c}}$ = \{v $\in$ V | c(v)=green \}\\
     D$^{\mathcal{I}_{\text{c}}}$ = 0\\
     Then $\mathcal{I}_{c}$ is a model for $\mathcal{T}$ and $\mathcal{A}_{G}$
but $\mathcal{T}_{c} \not \models q$ thus $\mathcal{T}$ and 
$\mathcal{A}_{G} \not \models q$ \\
     (<=) Assume that $(\mathcal{T},\mathcal{A_{G}}) \not \models q$ then
there is a model of $\mathcal{T}$ and $\mathcal{A_{G}}$ s.t. 
$\mathcal{I} \not \models q$. This obviously means that 
$D^{\mathcal{I}} = 0$. We define the mapping:\\
\begin{center}
$C_{\mathcal{I}}(v)$ = 
\begin{enumerate}
\item red if $v^{\mathcal{I}} \in R^{\mathcal{I}}$, otherwise
\item blue if $v^{\mathcal{I}} \in B^{\mathcal{I}}$, otherwise
\item green if $v^{\mathcal{I}} \in G^{\mathcal{I}}$
\end{enumerate}
\end{center}
Claim: $C_{\mathcal{I}}$ is a coloring of G: \\
     Let $(u,v) \in E$ and assume that $C_{\mathcal{I}}(u)$ = $C_{\mathcal{I}}(v)$ 
then $(u^{\mathcal{I}},v^{\mathcal{I}}) \in r^{\mathcal{I}}$ and 
$u^{\mathcal{I}}$ belongs to one of the concepts: \\
     R $\sqcap \exists$ r. R, B $\sqcap \exists$ r. B, G $\sqcap \exists$ r. G \\

\item Proposition 7.12  (data complexity of OMQA in $\mathcal{E}\mathcal{L}$) \\
\label{sec-1-2-2-2-3}
In $\mathcal{E}\mathcal{L}$ the query entailment problem for 
conjunctive query is P-hard w.r.t. data complexity.      
Proof: by LogSpace-reduction of path system acessibility.
A path system is of the form $P= (N,E,S,t)$ where
\begin{itemize}
\item N is a finite set of nodes. 
\item E $\subseteq N \times \times N \times N$ is an 
accessibility relation (we call its elements edges)
\item S $\subseteq N$ is a set of source nodes,
\item and $t \in N$ is a terminal node.
\end{itemize}

The set of accessible node of P is the smallest set of nodes s.t.
\begin{itemize}
\item every element of S is acessible.
\item if $n_{1}, n_{2}$ are accessible and $(n,n_{1}, n_{2})$ then n 
is accessible.\\
\end{itemize}
So given a path system P = (N,E,S,t), is t accessible?
Proof: (By LogSpace-reduction of path system accessibility)
The reduction:
(Data complexity implies $\mathcal{T}$ and $q$ shoulb be fixed)
$\mathcal{T} = \{\exists P_{1}. A \sqsubseteq B_{1}\}$
$q = A(x)$
$\mathcal{A} = \{ A(n)| n \in S\} \cup$ \\
      $\{P_{1}(e,j), P_{2}(e,k), P_{3}(n,e) | e=(n,j,k) \in E\}$
We have $(\mathcal{T},\mathcal{A}) \models A(t)$ 
iff t is accessible in P.
\end{enumerate}
\end{enumerate}
% Emacs 25.1.1 (Org mode 8.2.10)
\end{document}
